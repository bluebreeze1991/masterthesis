\chapter{绪论}
\section{研究背景和意义}
  由于环境因素和生活习惯的变化以及机体的内在因素等原因,许多人都患上了胃肠道疾病。例如,在2015年,在中国男性中常见癌症中分别有19\%和13\%由胃癌和食管癌构成,而在女性中胃癌和食管癌也分别占据了11\%和9\%的构成。其中我国男性和女性食道癌发病率分别为10万人中发病16.4人和10.2人,并且年平均死亡率为每10万人中死亡14.59人,为世界上食管癌死亡率最高的国家之一。\par
  医学内窥镜检查能够让医生直接观察人体内部胃肠道腔体的表现,并且该过程对病人仅产生轻微不适并且能在检查基础之上进行微创的取样、诊断和治疗,因此在现代医学中被广泛地应用于胃肠道疾病的诊断。如图1.1所示,医学内窥镜的前端包括了光源和镜头,医生通过将内窥镜前端伸入人体胃肠道,通过操纵调节纤维管中的金属丝来弯曲内窥镜的前端来使镜头转向不同的角度,镜头在胃肠道内采集到的图像通过纤维管中的光学纤维传到显示屏上供医生观察和诊断。借助于内窥镜技术,内科医生可以采集到高分辨率的医学内窥镜图像。\par
  另一方面,随着内窥镜技术的蓬勃发展和在现代医学诊疗中的广泛应用,越来越多的内窥镜图像在日常的内窥镜检查手术中产生。同时,胃肠道疾病本身表现多变并且胃肠道内部环境十分复杂,这使得实际的诊断效果十分依赖于内科医生的诊断经验与技术。有数据表明在首次内窥镜手术中发生的误诊比率依然高达26\%\cite{cheng2008clinical},因此开发计算机辅助诊断技术去帮助内科医生去发现和判别可疑的异常区域是十分具有意义的。
  
  
测试脚注\footnote{分别编号}。


\subsection{模板介绍1}

\subsubsection{模板测试}

\subsection{模板介绍2}

\section{系统要求}

\section{问题反馈}
测试脚注\footnote{脚注2}
